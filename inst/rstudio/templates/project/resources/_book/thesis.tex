% This is the University of GUelph LaTeX thesis template.
% Your comments and suggestions are more than welcome; please email
% them to seb@sciarra.io
%
% See https://www.reed.edu/cis/help/LaTeX/index.html for help. There are a
% great bunch of help pages there, with notes on
% getting started, bibtex, etc. Go there and read it if you're not
% already familiar with LaTeX.
%
% Any line that starts with a percent symbol is a comment.
% They won't show up in the document, and are useful for notes
% to yourself and explaining commands.
% Commenting also removes a line from the document;
% very handy for troubleshooting problems. -BTS

%%
%% Preamble
\documentclass[
12pt, % The default document font size, options: 10pt, 11pt, 12pt
twoside,
english]{guelphthesis}

%----------------------------------------------------------------------------------------
% PACKAGES
%----------------------------------------------------------------------------------------

\usepackage{tocloft} %needed for table of contents, list of figures, list of tables, list of appendices
\usepackage{graphicx,latexsym}
\usepackage{amsmath}
\usepackage{amssymb,amsthm}

\usepackage{longtable,booktabs,setspace}
\usepackage{lmodern}
\usepackage{float}
\usepackage{etoolbox}
\floatplacement{figure}{H}
% Thanks, @Xyv
\usepackage{calc}
% End of CII addition
\usepackage{rotating}
\usepackage{tocbibind} %includes list of figures, list of tables, and table of contents in table of contents
\usepackage{indentfirst} %needed so that first paragraph after each section titles has indent
\usepackage{lineno} %allows option for line numbering
\usepackage{draftwatermark} %for draft watermark
\SetWatermarkText{} %ensures draft is not printed when draft:false


% Syntax highlighting
  \usepackage{color}
  \usepackage{fancyvrb}
  \newcommand{\VerbBar}{|}
  \newcommand{\VERB}{\Verb[commandchars=\\\{\}]}
  \DefineVerbatimEnvironment{Highlighting}{Verbatim}{commandchars=\\\{\}}
  % Add ',fontsize=\small' for more characters per line
  \usepackage{framed}
  \definecolor{shadecolor}{RGB}{248,248,248}
  \newenvironment{Shaded}{\begin{snugshade}}{\end{snugshade}}
  \newcommand{\AlertTok}[1]{\textcolor[rgb]{0.94,0.16,0.16}{#1}}
  \newcommand{\AnnotationTok}[1]{\textcolor[rgb]{0.56,0.35,0.01}{\textbf{\textit{#1}}}}
  \newcommand{\AttributeTok}[1]{\textcolor[rgb]{0.13,0.29,0.53}{#1}}
  \newcommand{\BaseNTok}[1]{\textcolor[rgb]{0.00,0.00,0.81}{#1}}
  \newcommand{\BuiltInTok}[1]{#1}
  \newcommand{\CharTok}[1]{\textcolor[rgb]{0.31,0.60,0.02}{#1}}
  \newcommand{\CommentTok}[1]{\textcolor[rgb]{0.56,0.35,0.01}{\textit{#1}}}
  \newcommand{\CommentVarTok}[1]{\textcolor[rgb]{0.56,0.35,0.01}{\textbf{\textit{#1}}}}
  \newcommand{\ConstantTok}[1]{\textcolor[rgb]{0.56,0.35,0.01}{#1}}
  \newcommand{\ControlFlowTok}[1]{\textcolor[rgb]{0.13,0.29,0.53}{\textbf{#1}}}
  \newcommand{\DataTypeTok}[1]{\textcolor[rgb]{0.13,0.29,0.53}{#1}}
  \newcommand{\DecValTok}[1]{\textcolor[rgb]{0.00,0.00,0.81}{#1}}
  \newcommand{\DocumentationTok}[1]{\textcolor[rgb]{0.56,0.35,0.01}{\textbf{\textit{#1}}}}
  \newcommand{\ErrorTok}[1]{\textcolor[rgb]{0.64,0.00,0.00}{\textbf{#1}}}
  \newcommand{\ExtensionTok}[1]{#1}
  \newcommand{\FloatTok}[1]{\textcolor[rgb]{0.00,0.00,0.81}{#1}}
  \newcommand{\FunctionTok}[1]{\textcolor[rgb]{0.13,0.29,0.53}{\textbf{#1}}}
  \newcommand{\ImportTok}[1]{#1}
  \newcommand{\InformationTok}[1]{\textcolor[rgb]{0.56,0.35,0.01}{\textbf{\textit{#1}}}}
  \newcommand{\KeywordTok}[1]{\textcolor[rgb]{0.13,0.29,0.53}{\textbf{#1}}}
  \newcommand{\NormalTok}[1]{#1}
  \newcommand{\OperatorTok}[1]{\textcolor[rgb]{0.81,0.36,0.00}{\textbf{#1}}}
  \newcommand{\OtherTok}[1]{\textcolor[rgb]{0.56,0.35,0.01}{#1}}
  \newcommand{\PreprocessorTok}[1]{\textcolor[rgb]{0.56,0.35,0.01}{\textit{#1}}}
  \newcommand{\RegionMarkerTok}[1]{#1}
  \newcommand{\SpecialCharTok}[1]{\textcolor[rgb]{0.81,0.36,0.00}{\textbf{#1}}}
  \newcommand{\SpecialStringTok}[1]{\textcolor[rgb]{0.31,0.60,0.02}{#1}}
  \newcommand{\StringTok}[1]{\textcolor[rgb]{0.31,0.60,0.02}{#1}}
  \newcommand{\VariableTok}[1]{\textcolor[rgb]{0.00,0.00,0.00}{#1}}
  \newcommand{\VerbatimStringTok}[1]{\textcolor[rgb]{0.31,0.60,0.02}{#1}}
  \newcommand{\WarningTok}[1]{\textcolor[rgb]{0.56,0.35,0.01}{\textbf{\textit{#1}}}}


% To pass between YAML and LaTeX the dollar signs are added by CII
\title{Thesis title}
\author{Author name}
\year{year}
\date{date}
\advisor{advisor name}
\institution{University of Guelph}
\degree{degree}



\field{field (if necessary)}

\department{department}

  \let\cleardoublepage\clearpage


\urlstyle{rm}

%----------------------------------------------------------------------------------------
% CUSTOM COMMANDS
%----------------------------------------------------------------------------------------
%numbers lines before equations
%taken from https://tex.stackexchange.com/questions/43648/why-doesnt-lineno-number-a-paragraph-when-it-is-followed-by-an-align-equation
\newcommand*\patchAmsMathEnvironmentForLineno[1]{%
  \expandafter\let\csname old#1\expandafter\endcsname\csname #1\endcsname
  \expandafter\let\csname oldend#1\expandafter\endcsname\csname end#1\endcsname
  \renewenvironment{#1}%
     {\linenomath\csname old#1\endcsname}%
     {\csname oldend#1\endcsname\endlinenomath}}%
\newcommand*\patchBothAmsMathEnvironmentsForLineno[1]{%
  \patchAmsMathEnvironmentForLineno{#1}%
  \patchAmsMathEnvironmentForLineno{#1*}}%
\AtBeginDocument{%
\patchBothAmsMathEnvironmentsForLineno{equation}%
\patchBothAmsMathEnvironmentsForLineno{align}%
\patchBothAmsMathEnvironmentsForLineno{flalign}%
\patchBothAmsMathEnvironmentsForLineno{alignat}%
\patchBothAmsMathEnvironmentsForLineno{gather}%
\patchBothAmsMathEnvironmentsForLineno{multline}%
}


%nest all the \frontmatter functions in \oldfrontmatter, which allows us to redefine \frontmatter as everything it was with one modification to the
%draft watermark
\let\oldfrontmatter\frontmatter
%set page numbering to bottom center for \frontmatter
\fancypagestyle{frontmatter}{%
 \fancyhf{}% clear all header and footer fields
  \renewcommand{\headrulewidth}{0pt}
  \fancyhead[R]{\roman{page}}% Roman page number in footer centre

  }

\renewcommand{\frontmatter}{
  \oldfrontmatter
  
   %set page number font to Arial if ArialFont: false in YAML header
  
   \pagestyle{frontmatter} % add this to center page numbers
}

%set page numbering to bottom center for \mainmatter
\fancypagestyle{mainmatter}{%
 \fancyhf{}% clear all header and footer fields
  \renewcommand{\headrulewidth}{0pt}
  \fancyfoot[C]{\arabic{page}}% Roman page number in footer centre

   \hypersetup{pdfpagemode={UseOutlines},
    bookmarksopen=true,
    hypertexnames=true,
    colorlinks = true,
    allcolors = blue,
    %linkcolor = blue,
    %urlcolor= blue,
    %anchorcolor = blue,
    pdfstartview={FitV},
    breaklinks=true,
    hyperindex = true,
    backref=page}

  \cleardoublepage

  }

%nest all the \mainmatter functions in \oldmainmatter, which allows us to redefine \mainmatter as everything it was with one modification to the
%page numbering format
\newcommand{\setMainMatterLinespacing}{
 \setstretch{2} %default line spacing

  %change line spacing if specified in YAML header
        \setstretch{2}
  }

\let\oldmainmatter\mainmatter
\renewcommand{\mainmatter}{
  \oldmainmatter

  %change line spacing if specified in YAML header
  \setMainMatterLinespacing

  
  \pagestyle{mainmatter} % add this to center page numbers

}

%code below is important for linespacing to remain unaffected when kableExtra::landscape() is used andthe margin is specifically defined. Otherwise,
%linespacing for entire document goes to singlespacing for the text that follows the table.
\let\oldRestoreGeometry\restoregeometry
\renewcommand{\restoregeometry}{
  \oldRestoreGeometry

  %change line spacing if specified in YAML header
  \setMainMatterLinespacing
}

%change footnote and page number font to arial if desired




%----------------------------------------------------------------------------------------
%	TABLE OF CONTENTS, LIST OF FIGURES, & LIST OF TABLES
%----------------------------------------------------------------------------------------
%TABLE OF CONTENTS
\setlength{\cftbeforetoctitleskip}{0cm} %remove vertical space above table of contents

%two lines below ensure centered title for toc
%needed so that table of contents entry is not indented
\renewcommand{\contentsname}{Table of Contents} %change title for toc
\renewcommand{\cfttoctitlefont}{\hfill\fontsize{14}{14}\selectfont\bfseries\MakeUppercase}
\renewcommand{\cftaftertoctitle}{\hfill\hfill} %sometimes another \hfill is needed; depends on some setting in abovce code

%fonts for all entry level titles
\renewcommand\cftchapfont{\mdseries} %eliminate bolded chapter titles in toc
\renewcommand\cftsecfont{\mdseries} %eliminate bolded chapter titles in toc
\renewcommand\cftsubsecfont{\mdseries} %eliminate bolded chapter titles in toc
\renewcommand\cftsubsubsecfont{\mdseries} %eliminate bolded chapter titles in toc
\renewcommand\cftparafont{\mdseries} %eliminate bolded chapter titles in toc
\renewcommand\cftsubparafont{\mdseries} %eliminate bolded chapter titles in toc

%fonts for all entry level page numbers
\renewcommand{\cftchappagefont}{\mdseries} %remove bolding of page numbers for chapter headers in toc
\renewcommand\cftsecpagefont{\mdseries} %eliminate bolded chapter titles in toc
\renewcommand\cftsubsecpagefont{\mdseries} %eliminate bolded chapter titles in toc
\renewcommand\cftsubsubsecpagefont{\mdseries} %eliminate bolded chapter titles in toc
\renewcommand\cftparapagefont{\mdseries} %eliminate bolded chapter titles in toc
\renewcommand\cftsubparapagefont{\mdseries} %eliminate bolded chapter titles in toc

\renewcommand{\cftchapleader}{\cftdotfill{0.1}} %remove chapter bolding + modif dot spacing
\renewcommand{\cftdotsep}{0.1} %make dots in toc closer together

%spacing between toc items (should be all equal)
\setlength{\cftbeforechapskip}{0cm} %removes spacing before each chapter element
\renewcommand{\cftchapafterpnum}{\vskip6pt}
\renewcommand{\cftsecafterpnum}{\vskip6pt}
\renewcommand{\cftsubsecafterpnum}{\vskip6pt}
\renewcommand{\cftsubsubsecafterpnum}{\vskip6pt}
\renewcommand{\cftparaafterpnum}{\vskip6pt}
\renewcommand{\cftsubparaafterpnum}{\vskip6pt}

%remove header that appears in table of contents after first page
\renewcommand{\cftmarktoc}{}

%commands need to be redefined so that leading dots go all the way to the page numbers for all header levels (chap, sec, subsec, subsubsec, para, subpara
%%%general framework for commands below: cftXfillnum sets the format for the leading dots (\cftchapleader) and the page number (\cftchappagefont) such that leading dots proceed all the way to the page number with no spaces between dots and page number (\nobreak) at which wpoint paragraph mode ends (\par) and vertical spacing (defined  above) after item entry is inserted
%chapter (level 0)
\renewcommand{\cftchapfillnum}[1]{%
  {\cftchapleader}\nobreak
  {\cftchappagefont #1}\par\cftchapafterpnum
}

%sec (level 1)
\renewcommand{\cftsecfillnum}[1]{%
  {\cftsecleader}\nobreak
  {\cftsecpagefont #1}\par\cftsecafterpnum
}

%subsec (level 2)
\renewcommand{\cftsubsecfillnum}[1]{%
  {\cftsubsecleader}\nobreak
  {\cftsubsecpagefont #1}\par\cftsubsecafterpnum
}

%subsubsec (level 3)
\renewcommand{\cftsubsubsecfillnum}[1]{%
  {\cftsubsubsecleader}\nobreak
  {\cftsubsubsecpagefont #1}\par\cftsubsubsecafterpnum
}

%para (level 4)
\renewcommand{\cftparafillnum}[1]{%
  {\cftparaleader}\nobreak
  {\cftparapagefont #1}\par\cftparaafterpnum
}

%subpara (level 5)
\renewcommand{\cftsubparafillnum}[1]{%
  {\cftsubparaleader}\nobreak
  {\cftsubparapagefont #1}\par\cftsubparaafterpnum
}

%LIST OF TABLES
\renewcommand{\cfttabfont}{\mdseries} %set font for entries in lot
\renewcommand{\cfttabpagefont}{\mdseries} %set front for page numbers

\setlength{\cftbeforelottitleskip}{0cm} %remove vertical space above table of contents
\setlength{\cftafterlottitleskip}{0.5cm} %space between title for list of tables and list entries
%two lines below ensure centered title for toc
%needed so that table of contents entry is not indented
\renewcommand{\cftlottitlefont}{\hfill\fontsize{14}{14}\selectfont\bfseries\MakeUppercase}
\renewcommand{\cftafterlottitle}{\hfill} %sometimes another \hfill is needed; depends on some setting in abovce code

%commands need to be redefined so that leading dots go all the way to the page numbers for tables
%%%general framework for command below: cftfigfillnum sets the format for the leading dots (\cftfigleader) and the page number (\cftfigpagefont) such that leading dots proceed all the way to the page number with no spaces between dots and page number (\nobreak) at which point paragraph mode ends (\par) and vertical spacing (defined  below) after item entry is inserted
\setlength{\cftbeforetabskip}{0cm} %removes spacing before each chapter element
\renewcommand{\cfttabafterpnum}{\vskip6pt}

\renewcommand{\cfttabfillnum}[1]{%
  {\cfttableader}\nobreak
  {\cfttabpagefont #1}\par\cfttabafterpnum
}

%remove header that appears in list of tables after first page
\renewcommand{\cftmarklot}{}

%LIST OF FIGURES
\renewcommand{\cftfigfont}{\mdseries} %set font for entries in lof
\renewcommand{\cftfigpagefont}{\mdseries} %set front for page numbers

\setlength{\cftbeforeloftitleskip}{0cm} %remove vertical space above table of contents
\setlength{\cftafterloftitleskip}{0.5cm} %space between title for list of figures and list entries

%two lines below ensure centered title for toc
%needed so that table of contents entry is not indented
\renewcommand{\cftloftitlefont}{\hfill\fontsize{14}{14}\selectfont\bfseries\MakeUppercase}
\renewcommand{\cftafterloftitle}{\hfill} %sometimes another \hfill is needed; depends on some setting in abovce code

%commands need to be redefined so that leading dots go all the way to the page numbers for figures
%%%general framework for command below: cftfigfillnum sets the format for the leading dots (\cftfigleader) and the page number (\cftfigpagefont) such that leading dots proceed all the way to the page number with no spaces between dots and page number (\nobreak) at which wpoint paragraph mode ends (\par) and vertical spacing (defined  below) after item entry is inserted
\setlength{\cftbeforefigskip}{0cm} %removes spacing before each chapter element
\renewcommand{\cftfigafterpnum}{\vskip6pt}

\renewcommand{\cftfigfillnum}[1]{%
  {\cftfigleader}\nobreak
  {\cftfigpagefont #1}\par\cftfigafterpnum
}

%remove header that appears in list of figures after first page
\renewcommand{\cftmarklof}{}

%----------------------------------------------------------------------------------------
% LIST OF APPENDICES
%----------------------------------------------------------------------------------------
\newcommand{\listappname}{List of Appendices}
\newlistof[chapter]{app}{loa}{\listappname} %creates a new appendix counter that will be reset at the start of each \chapter

\setcounter{loadepth}{5} %loa will  go to depth of level 5
\setlength{\cftbeforeloatitleskip}{0cm} %remove vertical space above loa
\setlength{\cftafterloatitleskip}{0.5cm} %space between title for loa and list entries
\renewcommand{\cftmarkloa}{} %remove header titles

%two lines below ensure centered title for loa
%needed so that table of contents entry is not indented
\renewcommand{\cftloatitlefont}{\hfill\fontsize{14}{14}\selectfont\bfseries\MakeUppercase}
\renewcommand{\cftafterloatitle}{\hfill\hfill} %sometimes another \hfill is needed; depends on some setting in above code


%APPENDIX (level 0)
\renewcommand{\theapp}{\Alph{app}} %sets alphabetic counter for appendix
\renewcommand{\cftappfont}{\mdseries} %set font for level 0 entry in loa
\renewcommand{\cftapppagefont}{\mdseries} %set front for page numbers

\renewcommand{\cftapppresnum}{Appendix\space}
\renewcommand{\cftappaftersnum}{:\space}
\settowidth{\cftappnumwidth}{\cftapppresnum\theapp\cftappaftersnum\space}

\setlength{\cftbeforeappskip}{0cm} %removes vertical spacing before each chapter element
\renewcommand{\cftappafterpnum}{\vskip6pt}

%updates appendix counter, modifies chapter title such so that it is Appendix _letter_: #1

\newcommand{\app}[1]{%
  \refstepcounter{app}\pdfbookmark[-1]{\cftapppresnum\theapp\cftappaftersnum#1}{#1\theapp}%
  \chapter*{\fontsize{16}{16}\selectfont\bfseries\cftapppresnum\theapp\cftappaftersnum #1} %formats entry in document
  \addcontentsline{loa}{app}{{\cftapppresnum\theapp\cftappaftersnum}#1}%
  \par
}


% figure and table counting in appendix
\usepackage{chngcntr}


%leading dots for appendix (end immediately before page number)
\renewcommand{\cftappfillnum}[1]{%
 {\cftappleader}\nobreak{\cftapppagefont #1}\par\cftappafterpnum
}

%SECAPPENDIX (level 1; format A.1 : title)
\newlistentry[app]{secapp}{loa}{1}
\renewcommand{\thesecapp}{\theapp.\arabic{secapp}}
\renewcommand{\cftsecappfont}{\mdseries} %set font for level 1 entry in loa
\renewcommand{\cftsecapppagefont}{\mdseries} %set front for page numbers

\renewcommand{\cftsecapppresnum}{} %remove word 'Appendix'
\renewcommand{\cftsecappaftersnum}{\hspace{0.5cm}}  %replicate toc format for sub-level-0 headers \thesubappendix (i.e., A.1   title )

\setlength{\cftbeforesecappskip}{0cm} %removes vertical spacing before each chapter element
\renewcommand{\cftsecappafterpnum}{\vskip6pt}
\setlength{\cftsecappindent}{1.55em} %indentation in loa
\settowidth{\cftsecappnumwidth}{\cftsecapppresnum\thesecapp\cftsecappaftersnum\hspace{0.3cm}}

%updates appendix counter, modifies chapter title such so that it is Appendix _letter_: #1
\newcommand{\secapp}[1]{%
  \refstepcounter{secapp}\pdfbookmark[0]{#1}{#1\thesubapp}%
  \section*{\thesecapp\hspace{0.3cm} #1} %spacing between section number and title in text
  \addcontentsline{loa}{secapp}{{\thesecapp\cftsecappaftersnum}#1}%
  \par
}

%leading dots for appendix (end immediately before page number)
\renewcommand{\cftsecappfillnum}[1]{%
 {\cftsecappleader}\nobreak{\cftsecapppagefont #1}\par\cftsecappafterpnum
}


%SUBAPPENDIX (level 2; format A.1.1 : title)
\newlistentry[app]{subapp}{loa}{1}
\renewcommand{\thesubapp}{\thesecapp.\arabic{subapp}}
\renewcommand{\cftsubappfont}{\mdseries} %set font for level 2 entry in loa
\renewcommand{\cftsubapppagefont}{\mdseries} %set front for page numbers

\renewcommand{\cftsubapppresnum}{} %remove word 'Appendix'
\renewcommand{\cftsubappaftersnum}{\hspace{0.5cm}}  %replicate toc format for sub-level-0 headers \thesubappendix (i.e., A.1   title )

\setlength{\cftbeforesubappskip}{0cm} %removes vertical spacing before each chapter element
\renewcommand{\cftsubappafterpnum}{\vskip6pt}
\setlength{\cftsubappindent}{3.10em} %indentation in loa
%\renewcommand{\cftsubappnumwidth}{1.47cm}
\settowidth{\cftsubappnumwidth}{\thesubapp\cftsubappaftersnum\hspace{0.3cm}}

%updates appendix counter, modifies chapter title such so that it is Appendix _letter_: #1
\newcommand{\subapp}[1]{%
  \refstepcounter{subapp}\pdfbookmark[1]{#1}{#1\thesubapp}%
  \subsection*{\thesubapp\hspace{0.3cm} #1}%
  \addcontentsline{loa}{subapp}{{\thesubapp\cftsubappaftersnum}#1}%
  \par
}

%leading dots for appendix (end immediately before page number)
\renewcommand{\cftsubappfillnum}[1]{%
 {\cftsubappleader}\nobreak{\cftsubapppagefont #1}\par\cftsubappafterpnum
}


% SUBSUBAPPENDIX (level 3; format A.1.1.1  title)
\newlistentry[app]{subsubapp}{loa}{1}
\renewcommand{\thesubsubapp}{\thesubapp.\arabic{subsubapp}}
\renewcommand{\cftsubsubappfont}{\mdseries} %set font for level 3 entry in loa
\renewcommand{\cftsubsubapppagefont}{\mdseries} %set front for page numbers


\renewcommand{\cftsubsubapppresnum}{} %remove word 'Appendix'
\renewcommand{\cftsubsubappaftersnum}{\hspace{0.5cm}}  %space after subsubapp title

\setlength{\cftbeforesubsubappskip}{0cm} %removes vertical spacing before each chapter element
\renewcommand{\cftsubsubappafterpnum}{\vskip6pt}
\setlength{\cftsubsubappindent}{4.65em} %indentation in loa (1.55 *2)
\settowidth{\cftsubsubappnumwidth}{\thesubsubapp\cftsubsubappaftersnum\hspace{0.3cm}}

%updates appendix counter, modifies chapter title such so that it is Appendix _letter_: #1
\newcommand{\subsubapp}[1]{%
  \refstepcounter{subsubapp}\pdfbookmark[2]{#1}{#1\thesubsubapp}%
  \subsubsection*{\thesubsubapp\hspace{0.3cm} #1}%
  \addcontentsline{loa}{subsubapp}{{\thesubsubapp\cftsubsubappaftersnum}#1}%
  \par
}

%leading dots for appendix (end immediately before page number)
\renewcommand{\cftsubsubappfillnum}[1]{%
 {\cftsubsubappleader}\nobreak{\cftsubsubapppagefont #1}\par\cftsubsubappafterpnum
}

% PARA (level 4; format A.1.1.1.1  title)
\newlistentry[app]{paraapp}{loa}{1}
\renewcommand{\theparaapp}{\thesubsubapp.\arabic{paraapp}}
\renewcommand{\cftparaappfont}{\mdseries} %set font for level 4 entry in loa
\renewcommand{\cftparaapppagefont}{\mdseries} %set front for page numbers

\renewcommand{\cftparaapppresnum}{} %remove word 'Appendix'
\renewcommand{\cftparaappaftersnum}{\hspace{0.5cm}}  %space after paraapp title

\setlength{\cftbeforeparaappskip}{0cm} %removes vertical spacing before each chapter element
\renewcommand{\cftparaappafterpnum}{\vskip6pt}
\setlength{\cftparaappindent}{6.2em} %indentation in loa (1.55 *2)
\settowidth{\cftparaappnumwidth}{\theparaapp\cftparaappaftersnum\hspace{0.3cm}}

%updates appendix counter, modifies chapter title such so that it is Appendix _letter_: #1
\newcommand{\paraapp}[1]{%
  \refstepcounter{paraapp}\pdfbookmark[3]{#1}{#1\theparaapp}%
  \paragraph*{\theparaapp\hspace{0.3cm} #1}%
  \addcontentsline{loa}{paraapp}{{\theparaapp\cftparaappaftersnum}#1}%
  \par
}

%leading dots for appendix (end immediately before page number)
\renewcommand{\cftparaappfillnum}[1]{%
 {\cftparaappleader}\nobreak{\cftparaapppagefont #1}\par\cftparaappafterpnum
}

% SUBPARA (level 5; format A.1.1.1.1  title)
\newlistentry[app]{subparaapp}{loa}{1}
\renewcommand{\thesubparaapp}{\theparaapp.\arabic{subparaapp}}
\renewcommand{\cftsubparaappfont}{\mdseries} %set font for level 5 entry in loa
\renewcommand{\cftsubparaapppagefont}{\mdseries} %set front for page numbers

\renewcommand{\cftsubparaapppresnum}{} %remove word 'Appendix'
\renewcommand{\cftsubparaappaftersnum}{\hspace{0.5cm}}  %space after subparaapp title

\setlength{\cftbeforesubparaappskip}{0cm} %removes vertical spacing before each chapter element
\renewcommand{\cftsubparaappafterpnum}{\vskip6pt}
\setlength{\cftsubparaappindent}{7.75em} %indentation in loa (1.55 *2)
\settowidth{\cftsubparaappnumwidth}{\thesubparaapp\cftsubparaappaftersnum\hspace{0.3cm}}

%updates appendix counter, modifies chapter title such so that it is Appendix _letter_: #1
\newcommand{\subparaapp}[1]{%
  \refstepcounter{subparaapp}\pdfbookmark[4]{#1}{#1\thesubparaapp}%
  \paragraph*{\thesubparaapp\hspace{0.3cm} #1} %paragraph is used because subparagraph has weird numbering problem
  \addcontentsline{loa}{subparaapp}{{\thesubparaapp\cftsubparaappaftersnum}#1}%
  \par
}

%SUBSUBPARA (level 6; format A.1.1.1.1.1  title)
\newlistentry[app]{subsubparaapp}{loa}{1}
\renewcommand{\thesubsubparaapp}{\thesubparaapp.\arabic{subsubparaapp}}

\renewcommand{\cftsubsubparaapppresnum}{} %remove word 'Appendix'
\renewcommand{\cftsubsubparaappaftersnum}{\hspace{0.5cm}}  %space after subparaapp title

\setlength{\cftbeforesubsubparaappskip}{0cm} %removes vertical spacing before each chapter element
\renewcommand{\cftsubsubparaappafterpnum}{\vskip6pt}
\setlength{\cftsubsubparaappindent}{9.3em} %indentation in loa (1.55 *2)
\settowidth{\cftsubsubparaappnumwidth}{\thesubsubparaapp\cftsubsubparaappaftersnum\hspace{0.3cm}}

%updates appendix counter, modifies chapter title such so that it is Appendix _letter_: #1
\newcommand{\subsubparaapp}[1]{%
  \refstepcounter{subsubparaapp}\pdfbookmark[5]{#1}{#1\thesubsubparaapp}%
  \subparagraph*{\thesubsubparaapp\hspace{0.3cm} #1} %paragraph is used because subparagraph has weird numbering problem
  \addcontentsline{loa}{subsubparaapp}{{\thesubsubparaapp\cftsubsubparaappaftersnum}#1}%
  \par
}

%leading dots for appendix (end immediately before page number)
\renewcommand{\cftsubsubparaappfillnum}[1]{%
 {\cftsubsubparaappleader}\nobreak{\cftsubsubparaapppagefont #1}\par\cftsubsubparaappafterpnum
}

\newcommand{\listabbname}{List of Abbreviations}
\newlistof[chapter]{abb}{loab}{\listabbname} %creates a new appendix counter that will be reset at the start of each \chapter

\setlength{\cftbeforeloabtitleskip}{0cm} %remove vertical space above loab
\setlength{\cftafterloabtitleskip}{0.2cm} %space between title for loab and list entries

\renewcommand{\cftmarkloab}{} %remove header titles

%two lines below ensure centered title for loa
%needed so that table of contents entry is not indented
\renewcommand{\cftloabtitlefont}{\hfill\fontsize{14}{14}\selectfont\bfseries\MakeUppercase}
\renewcommand{\cftafterloabtitle}{\hfill\hfill} %sometimes another \hfill is needed; depends on some setting in above code



%----------------------------------------------------------------------------------------
% REFERENCES & HYPERLINKING
%----------------------------------------------------------------------------------------

\usepackage{hyperref}

\PassOptionsToPackage{backref=true}{biblatex}

\RequirePackage[autocite=inline, style = apa]{biblatex}
\addbibresource{bib/references.bib}


\DeclareSourcemap{\maps[datatype = bibtex]{\map{\step[fieldsource = journal, match = \regexp{\x{26}}, replace = \regexp{\{\\\x{26}\}}] }}}
\DeclareSourcemap{\maps[datatype = bibtex]{\map{\step[fieldsource = title, match = \regexp{\x{26}}, replace = \regexp{\{\\\x{26}\}}] }}}

\hypersetup{pdfpagemode={UseOutlines},
    bookmarksopen=true,
    backref=page}
\usepackage{hypernat}
%%adds escape character to ampersand characters in journal fields of .bib file
\DefineBibliographyStrings{english}{backrefpage={cited on p.},backrefpages={cited on pp.}}


\hypersetup{pdfpagemode={UseOutlines},
bookmarksopen=true, %allows bookmarks in pdf
hypertexnames=true, %enables counting when referencing to sections
colorlinks = true, % Set to true to enable coloring links, a nice option, false to turn them off
%citecolor = blue, % The color of citations
%linkcolor = blue, % The color of references to document elements (sections, figures, etc)
%urlcolor= blue,
%anchorcolor = blue, % The color of hyperlinks (URLs)
allcolors = blue,
pdfstartview={FitV},
breaklinks=true, backref=page
}



%example numbering
\newtheorem{theorem}{Theorem}[section]
\renewcommand{\thetheorem}{\theapp.\arabic{theorem}}
\newtheorem{example}{Example}
\renewcommand{\theexample}{\theapp.\arabic{example}}


%load additional latex packages needed within document
	\usepackage{booktabs}
\usepackage{longtable}
\usepackage{array}
\usepackage{multirow}
\usepackage{wrapfig}
\usepackage{float}
\usepackage{colortbl}
\usepackage{pdflscape}
\usepackage{tabu}
\usepackage{threeparttable}
\usepackage{threeparttablex}
\usepackage[normalem]{ulem}
\usepackage{makecell}
\usepackage{xcolor}

%----------------------------------------------------------------------------------------
% DOCUMENT OUTLINE
%----------------------------------------------------------------------------------------

% BEGIN DOCUMENT
\begin{document}
\frontmatter %pages will be numbered with roman numerals

  \maketitle

\setcounter{page}{2} %ensures abstract page number starts at roman numberal ii

\cleardoublepage
   %\thispagestyle{empty} %removes page number only for abstract page
  \begin{abstract}{2}{Despite the value that longitudinal research offers for understanding psychological processes, studies in organizational research rarely use longitudinal designs. One reason for the paucity of longitudinal designs may be the challenges they present for researchers. Three challenges of particular importance are that researchers have to determine 1) how many measurements to take, 2) how to space measurements, and 3) how to design studies when participants provide data with different response schedules (time unstructuredness). In systematically reviewing the simulation literature, I found that few studies comprehensively investigated the effects of measurement number, measurement spacing, and time structuredness (in addition to sample size) on model performance. As a consequence, researchers have little guidance when trying to conduct longitudinal research. To address these gaps in the literature, I conducted a series of simulation experiments. I found poor model performance across all measurement number/sample size pairings. That is, bias and precision were never concurrently optimized under any combination of manipulated variables. Bias was often low, however, with moderate measurement numbers and sample sizes. Although precision was frequently poor, the greatest improvements in precision resulted from using either seven measurements with \(N \ge 200\) or nine measurements with \(N \le 100\). With time-unstructured data, model performance systematically decreased across all measurement number/sample size pairings when the model incorrectly assumed an identical response pattern across all participants (i.e., time-structured data). Fortunately, when models were equipped to handle heterogeneous response patterns using definition variables, the poor model performance observed across all measurement number/sample size pairings no longer appeared. Altogether, the results of the current simulation experiments provide guidelines for researchers interested in modelling nonlinear change.}  %[linespacing][abstract]
  \end{abstract}

% notice how yaml variables are indexed with dollar signs and then passed into second argument of preambleItem environments
  \cleardoublepage
  \begin{preambleItem}{2}{Dedication}{{[}Dedication goes here if necessary{]}}
  \end{preambleItem}
  \cleardoublepage
   \begin{preambleItem}{2}{Acknowledgements}{{[}Acknowledgements go here{]}}
  \end{preambleItem}


%move page numbers to top right for list of tables, figures, and tables
\fancypagestyle{plain}{%
  \fancyhf{}% clear all header and footer fields
  \renewcommand{\headrulewidth}{0pt}
  \fancyhead[R]{\thepage}

   }

%table of contents
  \cleardoublepage
  \hypersetup{linkcolor = black, pdfborder= 0 0 0} %remove red borders around toc items
  \setcounter{secnumdepth}{5}
  \setcounter{tocdepth}{5}
  \tableofcontents
  \newpage

%list of tables
  \cleardoublepage
  \listoftables
  \newpage

%list of figures
  \cleardoublepage
  \listoffigures
  \newpage


%list of appendices
  \cleardoublepage
  \phantomsection
  \addcontentsline{toc}{chapter}{\listappname}
  \listofapp

  \newpage


\mainmatter % here the regular arabic numbering starts

\hypertarget{access-the-content-in-00-abstract.rmd-file-to-print-the-abstract.}{%
\chapter{access the content in 00-abstract.Rmd file to print the abstract.}\label{access-the-content-in-00-abstract.rmd-file-to-print-the-abstract.}}

Placeholder

\hypertarget{quotes}{%
\section{Quotes}\label{quotes}}

\hypertarget{footnotes}{%
\section{Footnotes}\label{footnotes}}

\hypertarget{figures}{%
\section{Figures}\label{figures}}

\hypertarget{tables}{%
\section{Tables}\label{tables}}

\hypertarget{equations-and-links}{%
\section{Equations and links}\label{equations-and-links}}

To generate the data, the \emph{multilevel logistic function} shown below in Equation \eqref{eq:logFunction-generation} was used:
\begin{align}
  y_{ij} = \uptheta_j + \frac{\upalpha_j - \uptheta_j}{{1 + e^\frac{\upbeta_j - time_i}{\upgamma_j}}} + \upepsilon_{ij}, 
\label{eq:logFunction-generation}
\end{align}
\noindent where \(\uptheta\) represents the baseline parameter, \(\upalpha\) represents the maximal elevation parameter, \(\upbeta\) represents the days-to-halfway elevation parameter, and \(\upgamma\) represents triquarter-halfway delta parameter. Note that, values for \(\uptheta\), \(\upalpha\), \(\upbeta\), and \(\upgamma\) were generated for each \emph{j} person across all \emph{i} time points, with an error value being randomly generated at each \emph{i} time point.

\hypertarget{Exp2}{%
\chapter{Experiment 2}\label{Exp2}}

In the \texttt{.Rmd} file, notice that a tag has been added so that a hyperlink to Experiment 2 can be done by using \texttt{{[}Experiment\ 2{]}(\#Exp2)} which produces \protect\hyperlink{Exp2}{Experiment 2}.

\newpage
\renewcommand\bibname{References}
\phantomsection
\addcontentsline{toc}{chapter}{References}
\printbibliography

%change numbering for figures, tables, and equations for appendices
\renewcommand\thefigure{\theapp.\arabic{figure}} %change figure numbering for appendix such that it goes A.1, A.2, etc.
\counterwithin{figure}{app} %reset figure number counter for each appendix

\renewcommand\thetable{\theapp.\arabic{table}} %change figure numbering for appendix such that it goes A.1, A.2, etc.
\counterwithin{table}{app} %reset figure number counter for each appendix

%reset equation number number counter for each appendix
\renewcommand{\theequation}{\theapp.\arabic{equation}}
\counterwithin{equation}{app} %reset figure number counter for each appendix

\counterwithin{chunk}{app} %reset code chunk numbering

\app{Ergodicity and the Need to Conduct Longitudinal Research}

\label{ergodicity}

To understand why cross-sectional results are unlikely to agree with longitudinal results for any given analysis, a discussion of data structures is apropos. Consider an example where a researcher obtains data from 50 people measured over 100 time points such that each row contains a \(p\) person's data over the 100 time points and each column contains data from 50 people at a \(t\) time point. For didactic purposes, all data are assumed to be sampled from a normal distribution. To understand whether findings in any given cross-sectional data set yield the same findings in any given longitudinal data set, the researcher randomly samples one cross-sectional and one longitudinal data set and computes the mean and variance in each set. To conduct a cross-sectional analysis, the researcher randomly samples the data across the 50 people at a given time point and computes a mean of the scores at the sampled time point (\(\bar{X}_t\)) using Equation \ref{eq:cross-mean} shown below:
\begin{align}
\bar{X}_t = \frac{1}{P}\sum^P_{p = 1} x_p,
\label{eq:cross-mean}
\end{align}
\hypertarget{example}{%
\section{Example}\label{example}}
\begin{example}
\protect\hypertarget{exm:taylor-estimates}{}\label{exm:taylor-estimates}Estimates of Taylor series approximation of \(f(x) = \cos(x)\) as the difference between the point of evaluation \(\mathrm{x}\) and the point of derivation \(\mathrm{a}\) increases.

\textup{Taylor series approximation of $\cos(x)$ (specifically, the second-order Taylor series; $P^2[\cos(x), a]$) estimates values that are exactly equal to the values returned by $\cos(x)$ when the point of evaluation (\textit{x}) is set to the point of derivation (\textit{a}). The example below computes the value predicted by the Taylor series approximation of $P^2[\cos(x), a]$ and by $\cos(x)$ when \textit{x} = \textit{a} = 0.}
\begin{align*}
P^2(\cos(x=0), a=0) &= \cos(x=0) \nonumber \\ 
1- \frac{1}{2}x^2 &=  \cos(0) \nonumber \\ 
1- \frac{1}{2}0^2 &=  1 \nonumber \\ 
1- 0 &=  1 \nonumber \\ 
1 &=  1 \nonumber \\ 
\end{align*}
\vspace*{-25mm}

\textup{Taylor series approximation of $\cos(x)$ (specifically, the second-order Taylor series; $P^2[\cos(x), a]$) estimates a value that is approximately equal ($\thickapprox$) to the value returned by $f\cos(x)$ when the difference between the point of evaluation \textit{x} and the point of derivation \textit{a} is small. The example below computes the value predicted by the Taylor series approximation of $P^2[\cos(x), a]$ and by $\cos(x)$ when \textit{x} = 1 and  \textit{a} = 0.}
\begin{align*}
P^2(\cos(x = 1), 0) &\thickapprox \cos(x = 1) \nonumber \\ 
1- \frac{1}{2}x^2 &\thickapprox   \cos(1) \nonumber \\ 
1- \frac{1}{2}1^2 &\thickapprox   0.54 \nonumber \\ 
1- 0.5 &\thickapprox   0.54 \nonumber \\ 
0.5 &\thickapprox 0.54 \nonumber \\ 
\end{align*}
\vspace*{-25mm}

\textup{Taylor series approximation of $f\cos(x)$ (specifically, the second-order Taylor series; $P^2[\cos(x), a]$) estimates a a value that is clearly not equal ($\neq$) to the value returned by $f\cos(x)$ when the difference between the point of evaluation \textit{x} and the point of derivation \textit{a} is large. The example below computes the value predicted by the Taylor series approximation of $P^2[\cos(x), a]$ and by $\cos(x)$ when \textit{x} = 4 and  \textit{a} = 0.}
\begin{align*}
P^2(\cos(x = 4), 0) &\neq \cos(x = 4) \nonumber \\ 
1- \frac{1}{2}x^2 &\neq  \cos(4) \nonumber \\ 
1- \frac{1}{2}4^2 &\neq  -0.65 \nonumber \\ 
1- 16 &\neq  -0.65 \nonumber \\ 
0.5 &\neq  -0.65 \nonumber \\ 
\end{align*}
\vspace*{-25mm}

\noindent \hrulefill
\end{example}
\hypertarget{code-block}{%
\section{Code block}\label{code-block}}

The code that I used to model logistic pattern of change (see \protect\hyperlink{data-generation}{data generation}) is shown in Code Block \ref{structured-model}. Note that, the code is largely excerpted from the \texttt{run\_exp\_simulations()} and \texttt{create\_logistic\_model\_ns()} functions from the \texttt{nonlinSims} package, and so readers interested in obtaining more information should consult the source code of this package. One important point to mention is that the model specified in Code Block \ref{structured-model} assumes time-structured data.
\begin{Shaded}
\begin{Highlighting}[numbers=left,,]
\CommentTok{\#Days on which measurements are assumed to be taken (note that model assumes time{-}structured data; that is, at each time point, participants provide data at the exact same moment). The measurement days obtained by finding the unique values in the \textasciigrave{}measurement\_day\textasciigrave{} column of the generated data set. }
\NormalTok{measurement\_days }\OtherTok{\textless{}{-}} \FunctionTok{unique}\NormalTok{(data}\SpecialCharTok{$}\NormalTok{measurement\_day) }

\CommentTok{\#Manifest variable names (i.e., names of columns containing data at each time point,}
\NormalTok{manifest\_vars }\OtherTok{\textless{}{-}}\NormalTok{ nonlinSims}\SpecialCharTok{:::}\FunctionTok{extract\_manifest\_var\_names}\NormalTok{(}\AttributeTok{data\_wide =}\NormalTok{ data\_wide)}

\CommentTok{\#Now convert data to wide format (needed for OpenMx)}
\NormalTok{data\_wide }\OtherTok{\textless{}{-}}\NormalTok{ data[ , }\FunctionTok{c}\NormalTok{(}\DecValTok{1}\SpecialCharTok{:}\DecValTok{3}\NormalTok{, }\DecValTok{5}\NormalTok{)] }\SpecialCharTok{\%\textgreater{}\%} 
    \FunctionTok{pivot\_wider}\NormalTok{(}\AttributeTok{names\_from =}\NormalTok{ measurement\_day, }\AttributeTok{values\_from =} \FunctionTok{c}\NormalTok{(obs\_score, actual\_measurement\_day))}
  
\CommentTok{\#Remove . from column names so that OpenMx does not run into error (this occurs because, with some spacing schedules, measurement days are not integer values.) }
\FunctionTok{names}\NormalTok{(data\_wide) }\OtherTok{\textless{}{-}} \FunctionTok{str\_replace}\NormalTok{(}\AttributeTok{string =} \FunctionTok{names}\NormalTok{(data\_wide), }\AttributeTok{pattern =} \StringTok{\textquotesingle{}}\SpecialCharTok{\textbackslash{}\textbackslash{}}\StringTok{.\textquotesingle{}}\NormalTok{, }\AttributeTok{replacement =} \StringTok{\textquotesingle{}\_\textquotesingle{}}\NormalTok{)}

\CommentTok{\#Latent variable names (theta = baseline, alpha = maximal elevation, beta = days{-}to{-}halfway elevation, gamma = triquarter{-}haflway elevation)}
\NormalTok{latent\_vars }\OtherTok{\textless{}{-}} \FunctionTok{c}\NormalTok{(}\StringTok{\textquotesingle{}theta\textquotesingle{}}\NormalTok{, }\StringTok{\textquotesingle{}alpha\textquotesingle{}}\NormalTok{, }\StringTok{\textquotesingle{}beta\textquotesingle{}}\NormalTok{, }\StringTok{\textquotesingle{}gamma\textquotesingle{}}\NormalTok{) }

\NormalTok{latent\_growth\_curve\_model }\OtherTok{\textless{}{-}} \FunctionTok{mxModel}\NormalTok{(}
  \AttributeTok{model =}\NormalTok{ model\_name,}
  \AttributeTok{type =} \StringTok{\textquotesingle{}RAM\textquotesingle{}}\NormalTok{, }\AttributeTok{independent =}\NormalTok{ T,}
  \FunctionTok{mxData}\NormalTok{(}\AttributeTok{observed =}\NormalTok{ data\_wide, }\AttributeTok{type =} \StringTok{\textquotesingle{}raw\textquotesingle{}}\NormalTok{),}
  
  \AttributeTok{manifestVars =}\NormalTok{ manifest\_vars,}
  \AttributeTok{latentVars =}\NormalTok{ latent\_vars,}
  
  \CommentTok{\#Residual variances; by using one label, they are assumed to all be equal (homogeneity of variance). That is, there is no complex error structure. }
  \FunctionTok{mxPath}\NormalTok{(}\AttributeTok{from =}\NormalTok{ manifest\_vars,}
         \AttributeTok{arrows=}\DecValTok{2}\NormalTok{, }\AttributeTok{free=}\ConstantTok{TRUE}\NormalTok{,  }\AttributeTok{labels=}\StringTok{\textquotesingle{}epsilon\textquotesingle{}}\NormalTok{, }\AttributeTok{values =} \DecValTok{1}\NormalTok{, }\AttributeTok{lbound =} \DecValTok{0}\NormalTok{),}
  
  \CommentTok{\#Latent variable covariances and variances (note that only the variances are estimated. )}
  \FunctionTok{mxPath}\NormalTok{(}\AttributeTok{from =}\NormalTok{ latent\_vars,}
         \AttributeTok{connect=}\StringTok{\textquotesingle{}unique.pairs\textquotesingle{}}\NormalTok{, }\AttributeTok{arrows=}\DecValTok{2}\NormalTok{,}
         \AttributeTok{free =} \FunctionTok{c}\NormalTok{(}\ConstantTok{TRUE}\NormalTok{,}\ConstantTok{FALSE}\NormalTok{, }\ConstantTok{FALSE}\NormalTok{, }\ConstantTok{FALSE}\NormalTok{, }
                  \ConstantTok{TRUE}\NormalTok{, }\ConstantTok{FALSE}\NormalTok{, }\ConstantTok{FALSE}\NormalTok{, }
                  \ConstantTok{TRUE}\NormalTok{, }\ConstantTok{FALSE}\NormalTok{, }
                  \ConstantTok{TRUE}\NormalTok{), }
         \AttributeTok{values=}\FunctionTok{c}\NormalTok{(}\DecValTok{1}\NormalTok{, }\ConstantTok{NA}\NormalTok{, }\ConstantTok{NA}\NormalTok{, }\ConstantTok{NA}\NormalTok{, }
                  \DecValTok{1}\NormalTok{, }\ConstantTok{NA}\NormalTok{, }\ConstantTok{NA}\NormalTok{, }
                  \DecValTok{1}\NormalTok{, }\ConstantTok{NA}\NormalTok{,}
                  \DecValTok{1}\NormalTok{),}
         \AttributeTok{labels=}\FunctionTok{c}\NormalTok{(}\StringTok{\textquotesingle{}theta\_rand\textquotesingle{}}\NormalTok{, }\StringTok{\textquotesingle{}NA(cov\_theta\_alpha)\textquotesingle{}}\NormalTok{, }\StringTok{\textquotesingle{}NA(cov\_theta\_beta)\textquotesingle{}}\NormalTok{, }
                  \StringTok{\textquotesingle{}NA(cov\_theta\_gamma)\textquotesingle{}}\NormalTok{,}
                  \StringTok{\textquotesingle{}alpha\_rand\textquotesingle{}}\NormalTok{,}\StringTok{\textquotesingle{}NA(cov\_alpha\_beta)\textquotesingle{}}\NormalTok{, }\StringTok{\textquotesingle{}NA(cov\_alpha\_gamma)\textquotesingle{}}\NormalTok{, }
                  \StringTok{\textquotesingle{}beta\_rand\textquotesingle{}}\NormalTok{, }\StringTok{\textquotesingle{}NA(cov\_beta\_gamma)\textquotesingle{}}\NormalTok{, }
                  \StringTok{\textquotesingle{}gamma\_rand\textquotesingle{}}\NormalTok{), }
         \AttributeTok{lbound =} \FunctionTok{c}\NormalTok{(}\FloatTok{1e{-}3}\NormalTok{, }\ConstantTok{NA}\NormalTok{, }\ConstantTok{NA}\NormalTok{, }\ConstantTok{NA}\NormalTok{, }
                    \FloatTok{1e{-}3}\NormalTok{, }\ConstantTok{NA}\NormalTok{, }\ConstantTok{NA}\NormalTok{, }
                    \DecValTok{1}\NormalTok{, }\ConstantTok{NA}\NormalTok{,}
                    \DecValTok{1}\NormalTok{), }
         \AttributeTok{ubound =} \FunctionTok{c}\NormalTok{(}\DecValTok{2}\NormalTok{, }\ConstantTok{NA}\NormalTok{, }\ConstantTok{NA}\NormalTok{, }\ConstantTok{NA}\NormalTok{, }
                    \DecValTok{2}\NormalTok{, }\ConstantTok{NA}\NormalTok{, }\ConstantTok{NA}\NormalTok{, }
                    \DecValTok{90}\SpecialCharTok{\^{}}\DecValTok{2}\NormalTok{, }\ConstantTok{NA}\NormalTok{, }
                    \DecValTok{45}\SpecialCharTok{\^{}}\DecValTok{2}\NormalTok{)),}
  
  \CommentTok{\# Latent variable means (linear parameters). Note that the parameters of beta and gamma do not have estimated means because they are nonlinear parameters (i.e., the logistic function\textquotesingle{}s first{-}order partial derivative with respect to each of those two parameters contains those two parameters. )}
  \FunctionTok{mxPath}\NormalTok{(}\AttributeTok{from =} \StringTok{\textquotesingle{}one\textquotesingle{}}\NormalTok{, }\AttributeTok{to =} \FunctionTok{c}\NormalTok{(}\StringTok{\textquotesingle{}theta\textquotesingle{}}\NormalTok{, }\StringTok{\textquotesingle{}alpha\textquotesingle{}}\NormalTok{), }\AttributeTok{free =} \FunctionTok{c}\NormalTok{(}\ConstantTok{TRUE}\NormalTok{, }\ConstantTok{TRUE}\NormalTok{), }\AttributeTok{arrows =} \DecValTok{1}\NormalTok{,}
         \AttributeTok{labels =} \FunctionTok{c}\NormalTok{(}\StringTok{\textquotesingle{}theta\_fixed\textquotesingle{}}\NormalTok{, }\StringTok{\textquotesingle{}alpha\_fixed\textquotesingle{}}\NormalTok{), }\AttributeTok{lbound =} \DecValTok{0}\NormalTok{, }\AttributeTok{ubound =} \DecValTok{7}\NormalTok{, }
         \AttributeTok{values =} \FunctionTok{c}\NormalTok{(}\DecValTok{1}\NormalTok{, }\DecValTok{1}\NormalTok{)),}
  
  \CommentTok{\#Functional constraints (needed to estimate mean values of fixed{-}effect parameters)}
  \FunctionTok{mxMatrix}\NormalTok{(}\AttributeTok{type =} \StringTok{\textquotesingle{}Full\textquotesingle{}}\NormalTok{, }\AttributeTok{nrow =} \FunctionTok{length}\NormalTok{(manifest\_vars), }\AttributeTok{ncol =} \DecValTok{1}\NormalTok{, }\AttributeTok{free =} \ConstantTok{TRUE}\NormalTok{, }
           \AttributeTok{labels =} \StringTok{\textquotesingle{}theta\_fixed\textquotesingle{}}\NormalTok{, }\AttributeTok{name =} \StringTok{\textquotesingle{}t\textquotesingle{}}\NormalTok{, }\AttributeTok{values =} \DecValTok{1}\NormalTok{, }\AttributeTok{lbound =} \DecValTok{0}\NormalTok{,  }\AttributeTok{ubound =} \DecValTok{7}\NormalTok{), }
  \FunctionTok{mxMatrix}\NormalTok{(}\AttributeTok{type =} \StringTok{\textquotesingle{}Full\textquotesingle{}}\NormalTok{, }\AttributeTok{nrow =} \FunctionTok{length}\NormalTok{(manifest\_vars), }\AttributeTok{ncol =} \DecValTok{1}\NormalTok{, }\AttributeTok{free =} \ConstantTok{TRUE}\NormalTok{, }
           \AttributeTok{labels =} \StringTok{\textquotesingle{}alpha\_fixed\textquotesingle{}}\NormalTok{, }\AttributeTok{name =} \StringTok{\textquotesingle{}a\textquotesingle{}}\NormalTok{, }\AttributeTok{values =} \DecValTok{1}\NormalTok{, }\AttributeTok{lbound =} \DecValTok{0}\NormalTok{,  }\AttributeTok{ubound =} \DecValTok{7}\NormalTok{), }
  \FunctionTok{mxMatrix}\NormalTok{(}\AttributeTok{type =} \StringTok{\textquotesingle{}Full\textquotesingle{}}\NormalTok{, }\AttributeTok{nrow =} \FunctionTok{length}\NormalTok{(manifest\_vars), }\AttributeTok{ncol =} \DecValTok{1}\NormalTok{, }\AttributeTok{free =} \ConstantTok{TRUE}\NormalTok{, }
           \AttributeTok{labels =} \StringTok{\textquotesingle{}beta\_fixed\textquotesingle{}}\NormalTok{, }\AttributeTok{name =} \StringTok{\textquotesingle{}b\textquotesingle{}}\NormalTok{, }\AttributeTok{values =} \DecValTok{1}\NormalTok{, }\AttributeTok{lbound =} \DecValTok{1}\NormalTok{, }\AttributeTok{ubound =} \DecValTok{360}\NormalTok{),}
  \FunctionTok{mxMatrix}\NormalTok{(}\AttributeTok{type =} \StringTok{\textquotesingle{}Full\textquotesingle{}}\NormalTok{, }\AttributeTok{nrow =} \FunctionTok{length}\NormalTok{(manifest\_vars), }\AttributeTok{ncol =} \DecValTok{1}\NormalTok{, }\AttributeTok{free =} \ConstantTok{TRUE}\NormalTok{, }
           \AttributeTok{labels =} \StringTok{\textquotesingle{}gamma\_fixed\textquotesingle{}}\NormalTok{, }\AttributeTok{name =} \StringTok{\textquotesingle{}g\textquotesingle{}}\NormalTok{, }\AttributeTok{values =} \DecValTok{1}\NormalTok{, }\AttributeTok{lbound =} \DecValTok{1}\NormalTok{, }\AttributeTok{ubound =} \DecValTok{360}\NormalTok{), }

  \FunctionTok{mxMatrix}\NormalTok{(}\AttributeTok{type =} \StringTok{\textquotesingle{}Full\textquotesingle{}}\NormalTok{, }\AttributeTok{nrow =} \FunctionTok{length}\NormalTok{(manifest\_vars), }\AttributeTok{ncol =} \DecValTok{1}\NormalTok{, }\AttributeTok{free =} \ConstantTok{FALSE}\NormalTok{, }
           \AttributeTok{values =}\NormalTok{ measurement\_days, }\AttributeTok{name =} \StringTok{\textquotesingle{}time\textquotesingle{}}\NormalTok{),}
  
  \CommentTok{\#Algebra specifying first{-}order partial derivatives; }
  \FunctionTok{mxAlgebra}\NormalTok{(}\AttributeTok{expression =} \DecValTok{1} \SpecialCharTok{{-}} \DecValTok{1}\SpecialCharTok{/}\NormalTok{(}\DecValTok{1} \SpecialCharTok{+} \FunctionTok{exp}\NormalTok{((b }\SpecialCharTok{{-}}\NormalTok{ time)}\SpecialCharTok{/}\NormalTok{g)), }\AttributeTok{name=}\StringTok{"Tl"}\NormalTok{),}
  \FunctionTok{mxAlgebra}\NormalTok{(}\AttributeTok{expression =} \DecValTok{1}\SpecialCharTok{/}\NormalTok{(}\DecValTok{1} \SpecialCharTok{+} \FunctionTok{exp}\NormalTok{((b }\SpecialCharTok{{-}}\NormalTok{ time)}\SpecialCharTok{/}\NormalTok{g)), }\AttributeTok{name =} \StringTok{\textquotesingle{}Al\textquotesingle{}}\NormalTok{), }
  
  \FunctionTok{mxAlgebra}\NormalTok{(}\AttributeTok{expression =} \SpecialCharTok{{-}}\NormalTok{((a }\SpecialCharTok{{-}}\NormalTok{ t) }\SpecialCharTok{*}\NormalTok{ (}\FunctionTok{exp}\NormalTok{((b }\SpecialCharTok{{-}}\NormalTok{ time)}\SpecialCharTok{/}\NormalTok{g) }\SpecialCharTok{*}\NormalTok{ (}\DecValTok{1}\SpecialCharTok{/}\NormalTok{g))}\SpecialCharTok{/}\NormalTok{(}\DecValTok{1} \SpecialCharTok{+} \FunctionTok{exp}\NormalTok{((b }\SpecialCharTok{{-}}\NormalTok{ time)}\SpecialCharTok{/}\NormalTok{g))}\SpecialCharTok{\^{}}\DecValTok{2}\NormalTok{), }\AttributeTok{name =} \StringTok{\textquotesingle{}Bl\textquotesingle{}}\NormalTok{),}
  \FunctionTok{mxAlgebra}\NormalTok{(}\AttributeTok{expression =}\NormalTok{  (a }\SpecialCharTok{{-}}\NormalTok{ t) }\SpecialCharTok{*}\NormalTok{ (}\FunctionTok{exp}\NormalTok{((b }\SpecialCharTok{{-}}\NormalTok{ time)}\SpecialCharTok{/}\NormalTok{g) }\SpecialCharTok{*}\NormalTok{ ((b }\SpecialCharTok{{-}}\NormalTok{ time)}\SpecialCharTok{/}\NormalTok{g}\SpecialCharTok{\^{}}\DecValTok{2}\NormalTok{))}\SpecialCharTok{/}\NormalTok{(}\DecValTok{1} \SpecialCharTok{+} \FunctionTok{exp}\NormalTok{((b }\SpecialCharTok{{-}}\NormalTok{time)}\SpecialCharTok{/}\NormalTok{g))}\SpecialCharTok{\^{}}\DecValTok{2}\NormalTok{, }\AttributeTok{name =} \StringTok{\textquotesingle{}Gl\textquotesingle{}}\NormalTok{),}
  
  \CommentTok{\#Factor loadings; all fixed and, importantly, constrained to change according to their partial derivatives (i.e., nonlinear functions) }
  \FunctionTok{mxPath}\NormalTok{(}\AttributeTok{from =} \StringTok{\textquotesingle{}theta\textquotesingle{}}\NormalTok{, }\AttributeTok{to =}\NormalTok{ manifest\_vars, }\AttributeTok{arrows=}\DecValTok{1}\NormalTok{, }\AttributeTok{free=}\ConstantTok{FALSE}\NormalTok{,  }
         \AttributeTok{labels =} \FunctionTok{sprintf}\NormalTok{(}\AttributeTok{fmt =} \StringTok{\textquotesingle{}Tl[\%d,1]\textquotesingle{}}\NormalTok{, }\DecValTok{1}\SpecialCharTok{:}\FunctionTok{length}\NormalTok{(manifest\_vars))),}
  \FunctionTok{mxPath}\NormalTok{(}\AttributeTok{from =} \StringTok{\textquotesingle{}alpha\textquotesingle{}}\NormalTok{, }\AttributeTok{to =}\NormalTok{ manifest\_vars, }\AttributeTok{arrows=}\DecValTok{1}\NormalTok{, }\AttributeTok{free=}\ConstantTok{FALSE}\NormalTok{,  }
         \AttributeTok{labels =} \FunctionTok{sprintf}\NormalTok{(}\AttributeTok{fmt =} \StringTok{\textquotesingle{}Al[\%d,1]\textquotesingle{}}\NormalTok{, }\DecValTok{1}\SpecialCharTok{:}\FunctionTok{length}\NormalTok{(manifest\_vars))), }
  \FunctionTok{mxPath}\NormalTok{(}\AttributeTok{from=}\StringTok{\textquotesingle{}beta\textquotesingle{}}\NormalTok{, }\AttributeTok{to =}\NormalTok{ manifest\_vars, }\AttributeTok{arrows=}\DecValTok{1}\NormalTok{,  }\AttributeTok{free=}\ConstantTok{FALSE}\NormalTok{,}
         \AttributeTok{labels =}  \FunctionTok{sprintf}\NormalTok{(}\AttributeTok{fmt =} \StringTok{\textquotesingle{}Bl[\%d,1]\textquotesingle{}}\NormalTok{, }\DecValTok{1}\SpecialCharTok{:}\FunctionTok{length}\NormalTok{(manifest\_vars))), }
  \FunctionTok{mxPath}\NormalTok{(}\AttributeTok{from=}\StringTok{\textquotesingle{}gamma\textquotesingle{}}\NormalTok{, }\AttributeTok{to =}\NormalTok{ manifest\_vars, }\AttributeTok{arrows=}\DecValTok{1}\NormalTok{,  }\AttributeTok{free=}\ConstantTok{FALSE}\NormalTok{,}
         \AttributeTok{labels =}  \FunctionTok{sprintf}\NormalTok{(}\AttributeTok{fmt =} \StringTok{\textquotesingle{}Gl[\%d,1]\textquotesingle{}}\NormalTok{, }\DecValTok{1}\SpecialCharTok{:}\FunctionTok{length}\NormalTok{(manifest\_vars))), }
  
  \CommentTok{\#Fit function used to estimate free parameter values. }
  \FunctionTok{mxFitFunctionML}\NormalTok{(}\AttributeTok{vector =} \ConstantTok{FALSE}\NormalTok{)}
\NormalTok{)}

\CommentTok{\#Use starting value function from OpenMx to generate good starting values (uses weighted least squares)}
\NormalTok{latent\_growth\_model }\OtherTok{\textless{}{-}} \FunctionTok{mxAutoStart}\NormalTok{(}\AttributeTok{model =}\NormalTok{ latent\_growth\_model)}

\CommentTok{\#Fit model using mxTryHard(). Increases probability of convergence by attempting model convergence by randomly shifting starting values. }
\NormalTok{model\_results }\OtherTok{\textless{}{-}} \FunctionTok{mxTryHard}\NormalTok{(latent\_growth\_model)}
\end{Highlighting}
\end{Shaded}
\hypertarget{multipage-table}{%
\section{Multipage table}\label{multipage-table}}



\end{document}
